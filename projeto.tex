\chapter{Desenvolvimento do Projeto}

O projeto do sistema de iluminação controlado pela internet foi desenvolvido com o objetivo de apresentar algumas funcionalidades, dadas as limitações, e demonstrar a potencialidade de economia de energia. As funcionalidades propostas foram as de, remotamente, ligar e desligar a iluminação; apresentar dois modos de controle da luminosidade: O modo manual no qual o usuário poderá escolher a intensidade da fita de LED e o modo automático no qual o sistema manterá um nível mínimo de luminosidade, caso as fontes de luz externas ao sistema apresentem intensidade menor que o nível escolhido pelo usuário para o modo automático. A demonstração da economia de energia do sistema é baseada na comparação das estimativas do consumo da iluminação com intensidade máxima e do consumo com intensidade controlada automaticamente.

\section{Montagem}

\section{Programação}

\section{Interfaces de Usuário}

\subsection{Aplicativo Móvel}

\subsection{Interface \textit{Web}}

\section{Teste Proposto}