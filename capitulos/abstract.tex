Led illumination holds some advantages on fluorescent and incandescent forms of illumination. LED lamps or strips are, in general more efficient, compact and have features like color variety when compared to those other forms of illumination and that's why they are utilized on public lighting, decoration or on small gadgets. Ease of controlling its intensity by fastly keying its power source is an important feature when it comes to the use of microcontrollers like those connected to the internet. This work demonstrates an implementation of a lighting system with a white LED strip controlled by the embedded system "ESP-8266"  by means of data exchange on the internet with a mobile application with the MQTT internet protocol and it has the capability of turning on, off, setting the led strip's intensity or even setting the automatic intensity mode. Measurements will be taken in order to demonstrate the energy saving capability of this proposed system. 

% Palavras-chave do resumo em Inglês
\begin{keywords}
    Illumination, LED, MQTT, ESP8266
\end{keywords}