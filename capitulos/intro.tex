\chapter{Introdução}

O termo "Internet das Coisas" \space vem sendo usado extensivamente nos últimos anos e isso se deve ao fato de esse termo estar entre alguns 
%em conjunto com outros como%
outros como "inteligência artificial", "\textit{Big Data}" \space e "computação em nuvem", que norteiam uma latente revolução industrial, da informação e até da tecnologia de forma geral. A possibilidade de ter dispositivos conectados à internet que podem ser controlados remotamente e podem gerar dados que são processados, analisados e usados para tornar processos cada vez mais eficientes e lucrativos tem movimentado um grande mercado em todo o mundo e promete continuar sua expansão. 

Existem certos padrões de comportamento do público consumidor e da indústria diante de  tecnologias emergentes que são caracterizados por descoberta e discussão inicial, difusão entre adotadores precoces e por fim, uso extensivo. Pesquisas indicam que as barreiras iniciais à adoção da "Internet das Coisas" \space vêm sendo quebradas e a nova tecnologia está sendo encontrada não apenas em novas empresas (\textit{startups}) mas também têm atraido atenção da indústria de grande porte \cite{forbes}. 
%referencia da forbes 
%www.forbes.com/sites/louiscolumbus/2018/08/16/iot-market-predicted-to-double-by-2021-reaching-520b/

Outra forte tendência é o uso de diodos emissores de luz (LED) com a finalidade de iluminação tanto para ambientes abertos quanto para ambientes fechados \cite{bndes}. Há alguns anos a tecnologia já era estudada mas devido a seu custo elevado, a sua adoção era inviável. O custo dos LEDs vem caindo graças à produção em larga escala e a melhorias na sua eficiência, dessa forma sua viabilidade, em comparação com lâmpadas fluorescentes e incandescentes, vêm aumentando já que o LED apresenta alta eficiência luminosa e longa vida útil apesar de ainda ser mais caro que as tecnologias concorrentes.

Outra importante vantagem dos LEDs é a sua capacidade de ter sua intensidade facilmente ajustada pela "dimmerização", isto é, rápido chaveamento de sua alimentação. Isto permite o controle rápido e fácil da intensidade luminosa permitindo um uso ainda mais eficiente.

\section{Objetivo}

O objetivo deste trabalho é projetar e implementar um sistema de iluminação com LEDs controlado pela internet por um aplicativo móvel ou página \textit{web}. Esse sistema deve ter as funcionalidades de ser ligado e desligado, ter sua intensidade ajustada manualmente pela internet e ainda ter um controle automático de intensidade. O sistema de iluminação contará com uma fita de LEDs, um sensor de luminosidade e um microcontrolador com comunicação WiFi para controlar o sistema. O protocolo de troca de mensagens será o MQTT que permitirá que o aplicativo e a página \textit{web} troquem mensagens com o sistema embarcado sobre a rede TCP/IP. O ajuste automático deve ser feito por um controle de malha fechada PID.

\section{Estrutura do Texto}

Neste capítulo foi feita uma contextualização acerca dos temas de "internet das coisas" \space e do uso de LEDs para iluminação. Os parágrafos que seguem apresentam o que é abordado nos próximos capítulos.

O Capítulo 2 apresenta os conceitos tratados no projeto e apresenta os temas abordados de forma clara e direta.

O Capítulo 3 descreve as funcionalidades do projetos que guiaram a escolha dos materiais usados e dos métodos empregados.

O Capítulo 4 demonstra, por meio de testes, as capacidades do sistema desenvolvido.

O Capítulo 5 apresenta as conclusões, experiências e sugestões para trabalhos futuros.