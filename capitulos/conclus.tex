\chapter{Conclusão}

Este trabalho descreveu a pesquisa e desenvolvimento de um sistema de iluminação controlado pela internet caracterizado por algumas funcionalidades fundamentais como o ligamento e desligamento remoto e ajuste manual ou automático de luminosidade. O sistema é composto por uma fita de LEDs brancos frios, e seu desligamento, ligamento e ajuste de luminosidade é controlado pelo chaveamento da sua alimentação pelo microcontrolador com comunicação WiFi. A interface do sistema com o usuário é feito pela internet por meio do protocolo de troca de mensagens MQTT, que opera sobre a camada de rede subjacente TCP/IP, por meio de um aplicativo móvel ou página da web.

Este trabalho foi, em essência, multidisciplinar ao tratar de temas como diodos emissores de luz, ergonomia da iluminação, internet das coisas, microcontroladores e teoria de controle. Dada a quantidade de temas, alguns deles não contaram com uma pesquisa aprofundada, mas todos eles contribuíram com o conhecimento e ferramentas necessários ao projeto de um sistema completo e funcional. Os assuntos que mais tiveram impacto no projeto tiveram abordagem mais ampla, como o protocolo MQTT que foi fundamental para a estrutura de comunicação simples e robusta sobre a internet. Outro assunto bastante pertinente ao trabalho foi a teoria de controle que forneceu o embasamento necessário para projetar o sistema de controle em malha fechada e as técnicas de ajuste e adequação do sistema a imprecisões e não-linearidades.

Os maiores desafios residiram no desenvolvimento do controle automático de luminosidade em função das fortes não-linearidades inerentes ao sensor, ao método de medição e ao ruído do sensor. Algumas técnicas foram usadas para reduzir o ruído do sensor, limitar comportamentos instáveis e dar mais robustez ao controle. 

Próximos trabalhos nessa área poderiam explorar o condicionamento e ajuste da cor da iluminação que interfere nos processos biológicos de pessoas expostas afetando conforto, níveis de stress e ciclos circadianos (padrões oscilatórios de hormônios e neurotransmissores ao longo do dia). Outra área que pode ser explorada é a implementação de novas funcionalidades como programações horárias do sistema de iluminação.