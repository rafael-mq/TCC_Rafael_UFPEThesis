%% Template para dissertação/tese na classe UFPEthesis
%% versão 0.9.2
%% (c) 2005 Paulo G. S. Fonseca
%% www.cin.ufpe.br/~paguso/ufpethesis

%% Carrega a classe ufpethesis
%% Opções: * Idiomas
%%           pt   - português (padrão)
%%           en   - inglês
%%         * Tipo do Texto
%%           bsc  - para monografias de graduação
%%           msc  - para dissertações de mestrado (padrão)
%%           qual - exame de qualificação doutorado
%%           prop - proposta de tese doutorado
%%           phd  - para teses de doutorado
%%         * Mídia
%%           scr  - para versão eletrônica (PDF) / consulte o guia do usuario
%%         * Estilo
%%           classic - estilo original à la TAOCP (deprecated)
%%           std     - novo estilo à la CUP (padrão)
%%         * Paginação
%%           oneside - para impressão em face única
%%           twoside - para impressão em frente e verso (padrão)
\documentclass[bsc]{ufpethesis}

%% Preâmbulo:
%% coloque aqui o seu preâmbulo LaTeX, i.e., declaração de pacotes,
%% (re)definições de macros, medidas, etc.

\usepackage{indentfirst}
\usepackage{setspace}
\usepackage{scrextend}
\addtokomafont{labelinglabel}{\sffamily}
\usepackage{suffix}
\usepackage[withpage]{acronym} % optional command: printonlyused

\onehalfspacing

%% Identificação:

% Universidade
% e.g. \university{Universidade de Campinas}
% Na UFPE, comente a linha a seguir
%%\university{Universidade Federal de Pernambuco}

% Endereço (cidade)
% e.g. \address{Campinas}
% Na UFPE, comente a linha a seguir
%%\address{<CIDADE DA IES>}

% Instituto ou Centro Acadêmico
% e.g. \institute{Centro de Ciências Exatas e da Natureza}
% Comente se não se aplicar
\institute{Centro de Tecnologia e Geociências}

% Departamento acadêmico
% e.g. \department{Departamento de Informática}
% Comente se não se aplicar
\department{Departamento de Eletrônica e Sistemas}

% Programa de pós-graduação
% e.g. \program{Pós-graduação em Ciência da Computação}
\program{Bacharelado em Engenharia Eletrônica}

% Área de titulação
% e.g. \majorfield{Ciência da Computação}
\majorfield{Engenharia Eletrônica}

% Título da dissertação/tese
% e.g. \title{Sobre a conjectura $P=NP$}
\title{Iluminação de LED controlada pela internet}

% Data da defesa
% e.g. \date{19 de fevereiro de 2003}
\date{<DATA DA DEFESA>}

% Autor
% e.g. \author{José da Silva}
\author{Rafael Moreira Queiroz}

% Orientador(a)
% Opção: [f] - para orientador do sexo feminino
% e.g. \adviser[f]{Profa. Dra. Maria Santos}
\adviser{Gilson Jerônimo Junior}

% Orientador(a)
% Opção: [f] - para orientador do sexo feminino
% e.g. \coadviser{Prof. Dr. Pedro Pedreira}
% Comente se não se aplicar
%\coadviser{NOME DO(DA) CO-ORIENTADOR(A)}

%% Inicio do documento
\begin{document}

%%
%% Parte pré-textual
%%
\frontmatter

% Folha de rosto
% Comente para ocultar
\frontpage

% Portada (apresentação)
% Comente para ocultar
\presentationpage

% Dedicatória
% Comente para ocultar
\begin{dedicatory}
    Dedico este trabalho a todos que o tornaram possível.
\end{dedicatory}

% Agradecimentos
% Se preferir, crie um arquivo à parte e o inclua via \include{}
\acknowledgements
    Agradeço primeiramente a Deus por sua graça e luz em minha vida, aos meus pais Adelmo e Aparecida e a minha irmã Laura por todo o apoio, amor e compreensão, à minha namorada Mirla por estar sempre ao meu lado frente a todas as dificuldades, aos meus queridos amigos Paulo, Felipe, Samuel e Matheus, e ao meu orientador Prof. Gilson Jerônimo Junior.

% Epígrafe
% Comente para ocultar
% e.g.
%  \begin{epigraph}[Tarde, 1919]{Olavo Bilac}
%  Última flor do Lácio, inculta e bela,\\
%  És, a um tempo, esplendor e sepultura;\\
%  Ouro nativo, que, na ganga impura,\\
%  A bruta mina entre os cascalhos vela.
%  \end{epigraph}
\begin{epigraph}{Abraham Lincoln}
Não sou obrigado a vencer, mas tenho o dever de ser verdadeiro. Não sou obrigado a ter sucesso, mas tenho o dever de corresponder à luz que tenho.
\end{epigraph}

% Resumo em Português
% Se preferir, crie um arquivo à parte e o inclua via \include{}
\resumo
A iluminação de LED apresenta algumas vantagens diante das tradicionais iluminações fluorescentes e incandescentes. Lâmpadas ou fitas de LED são em geral mais eficientes, compactas e oferecem vantagens como variedade de cores em comparação às variantes tradicionais de iluminação e por isso vêm sendo utilizadas tanto em iluminação pública como em decoração e em pequenos dispositivos. A facilidade de controlar sua intensidade por meio do rápido chaveamento de sua potência é outro atrativo, principalmente quando se trata do uso de microcontroladores conectados à internet. Este trabalho apresenta uma implementação de um sistema de iluminação com fita de LEDs brancos controlada pelo sistema embarcado "ESP-8266" por meio da troca de dados pela internet com um aplicativo móvel por meio do protocolo de internet "MQTT" com a capacidade de ligar, desligar, arbitrar a intensidade da fita no modo manual ou ter a intensidade controlada automaticamente. Serão feitas medições do consumo do sistema para demonstrar a capacidade de economia de energia do sistema proposto.

% Palavras-chave do resumo em Português
\begin{keywords}
    Iluminação, LED, MQTT, ESP8266
\end{keywords}

% Resumo em Inglês
% Se preferir, crie um arquivo à parte e o inclua via \include{}
\abstract
Led illumination holds some advantages on fluorescent and incandescent forms of illumination. LED lamps or strips are, in general more efficient, compact and have features like color variety when compared to those other forms of illumination and that's why they are utilized on public lighting, decoration or on small gadgets. Ease of controlling its intensity by fastly keying its power source is an important feature when it comes to the use of microcontrollers like those connected to the internet. This work demonstrates an implementation of a lighting system with a white LED strip controlled by the embedded system "ESP-8266"  by means of data exchange on the internet with a mobile application with the MQTT internet protocol and it has the capability of turning on, off, setting the led strip's intensity or even setting the automatic intensity mode. Measurements will be taken in order to demonstrate the energy saving capability of this proposed system. 
% Palavras-chave do resumo em Inglês
\begin{keywords}
    Illumination, LED, MQTT, ESP8266
\end{keywords}

% Sumário
% Comente para ocultar
\tableofcontents

% Lista de figuras
% Comente para ocultar
\listoffigures

% Lista de tabelas
% Comente para ocultar
\listoftables

\chapter*{Lista de Siglas}
\begin{acronym}[MQTT]
    \acro{IEEE}{\textit{Institute of Electrical and Electronics Engineers}}
    \acro{IoT}{\textit{Internet of Things}}
    \acro{IP}{\textit{Internet Protocol}}
    \acro{LED}{\textit{Light Emitting Diode}}
    \acro{MAC)}{\textit{Media Access Control}}
    \acro{MQTT}{\textit{Message Queue Telemetry Transport}}
    \acro{TCP}{\textit{Transmission Control Protocol}}
    \acro{WiFi}{\textit{Wireless Fidelity}}
\end{acronym}

%%
%% Parte textual
%%
\mainmatter

% É aconselhável criar cada capítulo em um arquivo à parte, digamos
% "capitulo1.tex", "capitulo2.tex", ... "capituloN.tex" e depois
% incluí-los com:
\chapter{Introdução}

sgfsfdgsdfhg
\chapter{Referencial Teórico}
\section{Iluminação de LED}
\subsection{LED}

\subsection{Ergonomia da Iluminação}

A ergonomia é o estudo científico das relações homem e máquina no ambiente de trabalho, visando segurança e eficiência na interação entre estes. A norma regulamentadora que trata de ergonomia, a NR 17, traz orientações sobre luminosidade no ambiente de trabalho com o objetivo de proteger a saúde física e psicológica do trabalhador. Algumas das orientações da norma técnica \cite{nr}:

\begin{labeling}{17.5.3.3}
    \item[17.5.3] Em todos os locais de trabalho deve haver iluminação adequada, natural ou artificial, geral ou suplementar, apropriada à natureza da atividade.\\
    \item[17.5.3.1]  A iluminação geral deve ser uniformemente distribuída e difusa.\\
    \item[17.5.3.2] A iluminação geral ou suplementar deve ser projetada e instalada de forma a evitar ofuscamento, reflexos incômodos, sombras e contrastes excessivos.\\
    \item[17.5.3.3] Os níveis mínimos de iluminamento a serem observados nos locais de trabalho são os valores de iluminâncias estabelecidos na NBR 5413, norma brasileira registrada no INMETRO.
\end{labeling}

A NBR ISO/CIE 8995-1  substituiu a NBR 5413 e especifica condições de iluminação para ambientes de trabalho internos e visa a eficiência na execução de tarefas visuais com eficiência, conforto e segurança durante todo o tempo de trabalho, seja esta iluminação natural, artificial ou uma combinação de ambas. A fim de atingir esse objetivo vários critérios são levados em conta, entre eles estão: aspectos da cor da luz, cintilação e iluminância.

Esta legislação estabelece os valores de iluminâncias médias mínimas em serviço para iluminação artificial em interiores, onde se realizem atividades de comércio, indústria, ensino, esporte e outras. Estes valores estão condicionados a função do trabalhador e tempo de determinada tarefa. A iluminância é uma grandeza de luminosidade representada pela letra E, sua unidade de medida é o “lux”, para medí-la utiliza-se o luxímetro, e é definida como  "limite da razão do fluxo luminoso recebido pela superfície em torno de um ponto considerado, para a área da superfície quando esta tende para o zero." (NBR 8995-1, 2013).  A iluminância pode impactar em como uma pessoa percebe e realiza uma tarefa visual, de forma que altos valores de iluminância causam desconforto e ofuscamento e baixos valores tornam o ambiente de trabalho sem estímulo e tedioso.

%A cintilação e efeitos estroboscópicos são causado pelo chaveamento frequente ou oscilação da iluminância em função de aspectos elétricos da iluminação e pode provocar distração e efeitos fisiológicos como dores de cabeça. Podem ainda levar a situações de risco pela alteração da percepção de máquinas de rotação sincronizadas com a frequência de oscilação da iluminância. Este efeito é análogo ao de ver, em filmagens, a roda de um carro em movimento que parece estar girando devagar ou estar parada; o mesmo pode acontecer em uma fábrica com uma máquina síncrona iluminada por uma fonte de luz alimentada pela mesma rede elétrica que alimenta a máquina.%

\section{Internet das Coisas}

Ao longo da história, vários pensadores, cientistas e inventores fizeram previsões convergentes para o futuro, previsões de que um dia o mundo teria uma rede interconectada de computadores e sensores. Hoje vemos essas previsões tomarem forma e podemos vislumbrar esse futuro imaginado por grandes mentes do passado. Nikolas Tesla disse certa vez em uma entrevista em 1926: “... o planeta inteiro se tornará um cérebro gigante, o que significa que todas as coisas serão partes reais e harmônicas de um todo… e os instrumentos que nos permitirão realizar isso serão incrivelmente mais simples que os telefones atuais. Qualquer homem será capaz de carregar um seu próprio bolso”. Já o primeiro dispositivo cotidiano a se conectar a internet foi uma torradeira criada por John Romkey em 1990 e que se conectava a um computador pela pilha de protocolos TCP/IP (\acl{TCP} / \acl{IP}) podendo ser ligada ou desligada pela internet.

Tecnologias correlatas a \ac{IoT} vêm sendo adotadas, nos últimos anos, por empresas de todos os portes e abrindo espaço para novos empreendimentos. Não há restrições claras para os limites de aplicações para \ac{IoT}; aliada a outra importante tecnologia que vem sendo extensivamente explorada, a inteligência artificial, a \ac{IoT} consegue prover vantagens como aumento de produtividade, extrema versatilidade na obtenção de dados e redução de despesas. Um exemplo interessante é a aplicação de dispositivos conectados à rede na produção do leite \cite{milk} na qual produtores têm colocado sensores nos calcanhares, pescoço, mamas e intestino das vacas para prever com exatidão os ciclos férteis dos animais, medir a qualidade do leite antes de extraí-lo e monitorar a saúde da vaca.

O mercado de \ac{IoT} pode ser separado em aplicações de consumo, da indústria e de serviços de infraestrutura da própria internet das coisas. O ramo voltado ao consumidor amplo é caracterizado principalmente por aplicações domiciliares como sistemas de segurança conectados e automação residencial, além de dispositivos vestíveis como monitores de sinais biológicos e relógios “inteligentes”. O setor industrial faz uso da internet das coisas em soluções de logística, agropecuária, sensoriamento em geral, planejamento urbano, monitoramento da rede elétrica, e vários outros exemplos. Por outro lado, para possibilitar que todos esse serviços de \ac{IoT} existam, há várias plataformas de infraestrutura de rede, sistemas embarcados, plataformas de software e serviços de armazenamento e análise de dados. Gigantes da tecnologia como “\textit{Google}”, “\textit{Microsoft}” e “\textit{Amazon}” têm setores totalmente voltado para \ac{IoT} e serviço de nuvem como o “\textit{Google Cloud}”, o “\textit{Microsoft Azure}” e o “\textit{Amazon Web Services}”.

\subsection{Redes de Computadores}

A comunhão dos computadores e das comunicações foi uma das principais revoluções tecnológicas modernas dando origem ao campo de redes de computadores com a demanda de organizar a comunicação entre computadores e criar sistemas computacionais para fazer essa comunicação.

Diz-se que uma rede de computadores é definida por dois ou mais computadores que, interligados por um meio físico, são capazes de trocar dados \cite{redes}. O conceito não especifica qual o meio físico que interliga os computadores que pode ser cabos metálicos, fibras óticas, micro-ondas, ondas de infravermelho; mas limita o meio a apenas um tipo, de forma que a “ampla rede mundial” (\textit{World Wide Web}) não é uma rede mas uma “rede de redes” e que a internet não é uma rede, e sim um sistema distribuído que, apesar de ser composto por vários computadores, é um sistema de softwares que opera sobre a rede e que dá a impressão ao usuário de ser um sistema coerente, uma rede não apresenta essa coerência aparente.

A \textit{World Wide Web} funciona de acordo com o modelo de cliente/servidor no qual um serviço cliente interage na rede por solicitações ao serviço servidor que responde a essas solicitações, podemos dizer também que o servidor provê recursos que os clientes consomem.

\subsection{A tecnologia \ac{WiFi}}

A popularização de computadores pessoais portáteis (notebooks) trouxe consigo a demanda que o acesso à internet se tornasse, também, móvel e assim foram desenvolvidas algumas arquiteturas de rede baseadas em ondas eletromagnéticas. A existência de diversos tipos de conexão sem fio (wireless) evidenciou a necessidade de uma padronização e assim o \ac{IEEE} lançou o padrão “802.11” para redes locais \cite{wifi} com uma série de especificação de controle de acesso ao meio (MAC) e camada física nas frequências de 900 MHz, 2.4, 3.6, 5 e 60 GHz com larguras de banda mais usuais de 20 ou 22 MHz e taxas de 1 até 3500 MB/s. Diversas revisões subsequentes do padrão como “a”, “b”, “g”, “n” especificam frequência, largura de banda, modulação entre outros fatores; a maioria dos dispositivos com internet sem fio é versátil para trabalhar com múltiplos modelos do padrão.
A \acf{WiFi} é uma marca da Wi-Fi Alliance, organização que promove a tecnologia e certifica produtos que obedecem a alguns requisitos de interoperabilidade do padrão \ac{IEEE} 802.11.

% ...
% \include{capituloN}



%%
%% Parte pós-textual
%%
\backmatter

% Apêndices
% Comente se não houver apêndices
%\appendix

% É aconselhável criar cada apêndice em um arquivo à parte, digamos
% "apendice1.tex", "apendice.tex", ... "apendiceM.tex" e depois
% incluí-los com:
% \chapter{Código Fonte}

\lstset{
    frame=tb,
    language=C++,
    aboveskip=3mm,
    belowskip=3mm,
    showstringspaces=false,
    columns=flexible,
    basicstyle={\small\ttfamily},
    numbers=left,                    
    numbersep=5pt,                  
    numberstyle=\tiny\color{gray},
    keywordstyle=\color{blue},
    commentstyle=\color{dkgreen},
    stringstyle=\color{mauve},
    breaklines=true,
    breakatwhitespace=true,
    tabsize=4,
    captionpos=b
}

\begin{lstlisting}

/************************** Libraries ***************************/
#include <Arduino.h>
#include <ESP8266WiFi.h>
#include "Adafruit_MQTT.h"
#include "Adafruit_MQTT_Client.h"

// Includes do Wifi Manager
#include <DNSServer.h>
#include <ESP8266WebServer.h>
#include <WiFiManager.h>         //https://github.com/tzapu/WiFiManager

/********************* WiFi Access Point *************************/

#define WLAN_SSID       "GVT-4238"
#define WLAN_PASS       "6703003118"

/********************* Adafruit.io Setup *************************/

#define AIO_SERVER      "io.adafruit.com"
#define AIO_SERVERPORT  1883
#define AIO_USERNAME    "RafaelMQ"
#define AIO_KEY         "f6ed222e222c446088110578b7bd0146"

/********************* PIN DEFINITIONS **************************/
#define USER_LED D1

#define pwm_map_n(x) round(map(x, 0, 100, 0, 1023))

/************** GLOBAL STATE (don't change this!) ***************/

// Create an ESP8266 WiFiClient class to connect to the MQTT server.
WiFiClient client;

// Setup the MQTT client class by passing in the WiFi client and MQTT server and login details.
Adafruit_MQTT_Client mqtt(&client, AIO_SERVER, AIO_SERVERPORT, AIO_USERNAME, AIO_USERNAME, AIO_KEY);

/************************ Feeds ********************************/

Adafruit_MQTT_Subscribe timefeed = Adafruit_MQTT_Subscribe(&mqtt, "time/seconds");

// Setup a feed called 'pwmin' for subscribing to changes on the slider
Adafruit_MQTT_Subscribe pwmin = Adafruit_MQTT_Subscribe(&mqtt, AIO_USERNAME "/feeds/pwmin");

// Setup a feed called 'onoff' for subscribing to changes to the button
Adafruit_MQTT_Subscribe onoffbutton = Adafruit_MQTT_Subscribe(&mqtt, AIO_USERNAME "/feeds/onoff");

Adafruit_MQTT_Publish pwmout = Adafruit_MQTT_Publish(&mqtt, AIO_USERNAME "/feeds/pwmout");

/********************* Variables *****************************/
int sec;
int minute;
int hour;

int timeZone = -3; // utc-3 eastern daylight time (brasilia)

int global_pwm = 100;
bool led_state = false;

\end{lstlisting}
% \include{apendice2}
% ...
% \include{apendiceM}


% Bibliografia
% É aconselhável utilizar o BibTeX a partir de um arquivo, digamos "biblio.bib".
% Para ajuda na criação do arquivo .bib e utilização do BibTeX, recorra ao
% BibTeXpress em www.cin.ufpe.br/~paguso/bibtexpress
\nocite{*}
\bibliographystyle{alpha}
\bibliography{sample}

% Cólofon
% Inclui uma pequena nota com referência à UFPEThesis
% Comente para omitir
\colophon

%% Fim do documento
\end{document}
